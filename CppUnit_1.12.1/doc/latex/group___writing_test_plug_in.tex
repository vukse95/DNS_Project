\hypertarget{group___writing_test_plug_in}{
\section{Writing Test Plug-\/in}
\label{group___writing_test_plug_in}\index{Writing Test Plug-\/in@{Writing Test Plug-\/in}}
}
Creating a test plug-\/in is really simple:
\begin{DoxyItemize}
\item make your project a dynamic library (with VC++, choose Win32 Dynamic Library in the project wizard), and link against the dynamic library version of CppUnit (cppunit$\ast$\_\-dll.lib for VC++).
\item in a cpp file, include TestPlugIn.h, and use the macro CPPUNIT\_\-PLUGIN\_\-IMPLEMENT() to declare the test plug-\/in.
\item That's it, you're done! All the tests registered using the TestFactoryRegistry, CPPUNIT\_\-TEST\_\-SUITE\_\-NAMED\_\-REGISTRATION, or CPPUNIT\_\-TEST\_\-SUITE\_\-REGISTRATION will be visible to other plug-\/in and to the DllPlugInRunner.
\end{DoxyItemize}

Example: 
\begin{DoxyCode}
 #include <cppunit/include/plugin/TestPlugIn.h>

 CPPUNIT_PLUGIN_IMPLEMENT();
\end{DoxyCode}


The interface CppUnitTestPlugIn is automatically implemented by the previous macro. You can define your own implementation.

To provide your custom implementation of the plug-\/in interface, you must:
\begin{DoxyItemize}
\item create a class that implements the CppUnitTestPlugIn interface
\item use CPPUNIT\_\-PLUGIN\_\-EXPORTED\_\-FUNCTION\_\-IMPL() with your class to export the plug-\/in interface
\item implements the 'main' function with CPPUNIT\_\-PLUGIN\_\-IMPLEMENT\_\-MAIN().
\end{DoxyItemize}

Some of the reason you may want to do this:
\begin{DoxyItemize}
\item You do not use the TestFactoryRegistry to register your test.
\item You want to create a custom listener to use with DllPlugInRunner.
\item You want to do initialize some globale resources before running the test (setting up COM for example).
\end{DoxyItemize}

See CppUnitTestPlugIn for further detail on how to do this.

Creating your own test plug-\/in with VC++:
\begin{DoxyItemize}
\item Create a new \char`\"{}Win32 Dynamic Library\char`\"{} project, choose the empty template
\item For the Debug Configuration, add cppunitd\_\-dll.lib to 'Project Settings/Link/Object/Libariries modules', and for the Release Configuration, add cppunit\_\-dll.lib.
\item For All Configuration, in 'C++/Preprocessor/Preprocessors definitions', add the symbol 'CPPUNIT\_\-DLL' at the end of the line (it means that you are linking against cppunit dll).
\item Create a 'main' file that contains: \begin{DoxyVerb}
#include <cppunit/plugin/TestPlugIn.h>

CPPUNIT_PLUGIN_IMPLEMENT();\end{DoxyVerb}

\item Add your tests
\item You're done !
\end{DoxyItemize}

See examples/simple/simple\_\-plugin.dsp for an example.

Notes to VC++ users:
\begin{DoxyItemize}
\item you can run a post-\/build check on the plug-\/in. Add the following line to your post-\/build tab: \char`\"{}DllPlugInTesterd\_\-dll.exe \$(TargetPath)\char`\"{}. DllPlugInTesterd\_\-dll.exe need to be some place were it can be found (path, ...), or you need to indicate the correct path.  is the filename of your plug-\/in.
\item you can debug your DLL, set the executable for debug session to the plug-\/in runner, and the name of the DLL in the program arguments ( won't work this time).
\end{DoxyItemize}

How does it works ?

When CppUnit is linked as a DLL, the singleton used for the TestFactoryRegistry is the same for the plug-\/in runner (also linked against CppUnit DLL). This means that the tests registered with the macros (at static initialization) are registered in the same registry. As soon as a DLL is loaded by the PlugInManager, the DLL static variable are constructed and the test registered to the TestFactoryRegistry.

After loading the DLL, the PlugInManager look-\/up a specific function exported by the DLL. That function returns a pointer on the plug-\/in interface, which is later used by the PlugInManager.

\begin{DoxySeeAlso}{See also}
\hyperlink{group___creating_test_suite}{Creating TestSuite}. 
\end{DoxySeeAlso}
